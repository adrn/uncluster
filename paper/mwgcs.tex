\documentclass[manuscript, letterpaper]{aastex6}

\include{gitstuff}
% ----------------------------------- %
% start of AASTeX mods by DWH and DFM %
% ----------------------------------- %

\setlength{\voffset}{0in}
\setlength{\hoffset}{0in}
\setlength{\textwidth}{6in}
\setlength{\textheight}{9in}
\setlength{\headheight}{0ex}
\setlength{\headsep}{\baselinestretch\baselineskip} % this is 2 lines in ``manuscript''
\setlength{\footnotesep}{0in}
\setlength{\topmargin}{-\headsep}
\setlength{\oddsidemargin}{0.25in}
\setlength{\evensidemargin}{0.25in}

\linespread{0.54} % close to 10/13 spacing in ``manuscript''
\setlength{\parindent}{0.54\baselineskip}
\hypersetup{colorlinks = false}
\makeatletter % you know you are living your life wrong when you need to do this
\long\def\frontmatter@title@above{
\vspace*{-\headsep}\vspace*{\headheight}
\noindent\footnotesize
{\noindent\footnotesize\textsc{\@journalinfo}}\par
{\noindent\scriptsize Preprint typeset using \LaTeX\ style AASTeX6
with modifications by DWH and DFM.
}\par\vspace*{-\baselineskip}\vspace*{0.625in}
}%
\long\def\frontmatter@abstractheading{%
\makeaffils
  \vspace*{-\baselineskip}\vspace*{1.5pt}
  \vspace*{0.13189in}
 \begingroup
  \centering
  \abstractname
  \vskip 1mm
  \par
 \endgroup
 \everypar{\rightskip=0.0in\leftskip=\rightskip}\par
}%
\def\frontmatter@keys@format{\vspace*{0.5mm}%
  \settowidth{\keys@width}{\normalsize\@keys@name}%
  \rightskip=0.0in\leftskip=\rightskip\parindent=0pt%
    \hangindent=\keys@width\hangafter=1\normalsize\raggedright}%
\def\twodigits#1{\ifnum#1<10 0\fi\the#1}
\def\mydate{\leavevmode\hbox{\the\year-\twodigits\month-\twodigits\day}}
\makeatother
\renewcommand{\today}{\mydate}

% Section spacing:
\makeatletter
\let\origsection\section
\renewcommand\section{\@ifstar{\starsection}{\nostarsection}}
\newcommand\nostarsection[1]{\sectionprelude\origsection{#1}}
\newcommand\starsection[1]{\sectionprelude\origsection*{#1}}
\newcommand\sectionprelude{\vspace{1em}}
\let\origsubsection\subsection
\renewcommand\subsection{\@ifstar{\starsubsection}{\nostarsubsection}}
\newcommand\nostarsubsection[1]{\subsectionprelude\origsubsection{#1}}
\newcommand\starsubsection[1]{\subsectionprelude\origsubsection*{#1}}
\newcommand\subsectionprelude{\vspace{1em}}
\makeatother

\widowpenalty=10000
\clubpenalty=10000

\sloppy\sloppypar

% ------------------ %
% end of AASTeX mods %
% ------------------ %

% Load common packages

\usepackage{amsmath}
\usepackage{amsfonts}
\usepackage{amssymb}
\usepackage{booktabs}

\usepackage{graphicx}
\usepackage{color}

\usepackage{hyperref}
\definecolor{linkcolor}{rgb}{0,0,0.2}
\hypersetup{colorlinks=true,linkcolor=linkcolor,citecolor=linkcolor,
            filecolor=linkcolor,urlcolor=linkcolor}
\hypersetup{pageanchor=false}

\newcommand{\documentname}{\textsl{Article}}
\newcommand{\sectionname}{Section}
\newcommand{\figname}{Figure}
\newcommand{\eqname}{Equation}
\newcommand{\tblname}{Table}

% Packages / projects / programming
\newcommand{\package}[1]{\textsl{#1}}
\newcommand{\acronym}[1]{{\small{#1}}}
\newcommand{\project}[1]{\package{#1}}
\newcommand{\rewinder}{\project{Rewinder}}
\newcommand{\streakline}{\project{Streakline}}
\newcommand{\superfreq}{\project{SuperFreq}}

\newcommand{\github}{\project{GitHub}}
\newcommand{\python}{\project{Python}}

% For referee
\newcommand{\changes}[1]{{\color{red} #1}}

% Stats / probability
\newcommand{\given}{\,|\,}
\newcommand{\norm}{\mathcal{N}}

% Maths
\newcommand{\dd}{\mathrm{d}}
\newcommand{\transpose}[1]{{#1}^{\mathsf{T}}}
\newcommand{\inverse}[1]{{#1}^{-1}}
\newcommand{\argmin}{\operatornamewithlimits{argmin}}
\newcommand{\mean}[1]{\left< #1 \right>}

% Unit shortcuts
\newcommand{\msun}{\ensuremath{\mathrm{M}_\odot}}
\newcommand{\kms}{\ensuremath{\mathrm{km}~\mathrm{s}^{-1}}}
\newcommand{\pc}{\ensuremath{\mathrm{pc}}}
\newcommand{\kpc}{\ensuremath{\mathrm{kpc}}}
\newcommand{\kmskpc}{\ensuremath{\mathrm{km}~\mathrm{s}^{-1}~\mathrm{kpc}^{-1}}}

% Misc.
\newcommand{\bs}[1]{\boldsymbol{#1}}

% Astronomy
\newcommand{\DM}{{\rm DM}}
\newcommand{\feh}{\ensuremath{{[{\rm Fe}/{\rm H}]}}}
\newcommand{\df}{\acronym{DF}}

% TO DO
\newcommand{\todo}[1]{{\color{red} TODO: #1}}


\begin{document}

\title{The stellar halo of the Milky Way and globular cluster debris I: predictions for kinematic structure}
\author{Adrian~M.~Price-Whelan\altaffilmark{\pu,\adrn},
        Oleg Gnedin\altaffilmark{\umich}
%        Kathryn V. Johnston\altaffilmark{\colum}
%        David N. Spergel\altaffilmark{\cca,\pu}
%       Hans-Walter~Rix\altaffilmark{\mpia}
}

% Affiliations
\newcommand{\pu}{1}
\newcommand{\adrn}{2}
\newcommand{\umich}{3}
% \newcommand{\colum}{4}
% \newcommand{\cca}{5}
%\newcommand{\mpia}{6}

\altaffiltext{\pu}{Department of Astrophysical Sciences,
                   Princeton University, Princeton, NJ 08544, USA}
\altaffiltext{\adrn}{To whom correspondence should be addressed:
                     \texttt{adrn@princeton.edu}}

% TODO: michigan
\altaffiltext{\umich}{Department of Astronomy ??,
                      University of Michigan,
                      USA}
% \altaffiltext{\colum}{Department of Astronomy,
%                       Columbia University,
%                       550 W 120th St.,
%                       New York, NY 10027, USA}
% TODO: Simons
% \altaffiltext{\cca}{Simons Center for Computational Astrophysics,
%                       New York, NY XXX, USA}
%\altaffiltext{\mpia}{Max-Planck-Institut f\"ur Astronomie,
%                    K\"onigstuhl 17, D-69117 Heidelberg, Germany}

\begin{abstract}
% Context
Milky Way halo globular clusters probably accreted -- opportunity to study accretion history, blah blah.


Recent work studying the detailed chemical abundances of Milky Way halo stars
and the kinematic evolution of globular clusters suggest that (1) a significant
fraction of the stellar halo around the Milky Way could be composed of disrupted
globular clusters, and, consequently, that (2) the present globular cluster
population may be a small fraction ($<$5\%) of the initial population.
These suggestions make strong predictions about the structure---in kinematics
and chemistry---of halo stars that depend strongly on the origins of the initial
globular cluster system.
% Aims
Here we generate a set of mock stellar halos formed from disrupted globular
clusters with a range of formation scenarios that all broadly reproduce the
present-day spatial distribution and mass function of the remnant cluster
population.
We use these halos to predict the kinematic clustering of halo stars and the
expected number of thin stellar streams around the Milky Way for each of the
explored scenarios.
% Methods
We consider two main classes of initial conditions: (1) all accreted, in which
all globular clusters were deposited early on through mergers with other
galaxies, and (2) all formed in situ, in which all globular clusters formed
early on in the Galaxy.
For each, we consider cluster orbits drawn from isotropic, radially-anisotropic,
and tangentially-anisotropic distribution functions.
% Results
We find that the end-of-Gaia kinematic measurements for XX stars within XX kpc
of the Sun will distinguish the scenarios considered in this work.
We also predict that within XX kpc of the Galactic center, there are XX thin,
recently fully-disrupted globular-cluster streams.
Of these, XX have high enough surface brightness to have already been detected,
in agreement with the number of thin streams discovered thus far.

\end{abstract}

\keywords{
  Galaxy: halo
  ---
  globular clusters: general
  ---
  stars: kinematics and dynamics
  ---
  Galaxy: structure
  ---
  Galaxy: kinematics and dynamics
}

\section{Introduction}\label{sec:introduction}

with 10000 GCs, surface brightness fluctuations at z=2?!

Explain GC population of MW,

\section{Methods}\label{sec:methods}

The Milky Way hosts at least XX metal-poor globular clusters (GCs) at
Galactocentric radii between $XX < r < YY~\kpc$ that were likely accreted from
tidally-disrupted satellite galaxies;
the mass and spherically-averaged density distribution of the known GCs is shown
in \figname~\ref{xx}.
Some number of GCs have also been destroyed by a combination of the tidal field
of the Galaxy and internal relaxation processes.
Our goal is to reproduce the properties of the present-day GC population with
different models for the initial GC population, varying the total mass in GCs
(the initial number, at fixed mass distribution) and the dynamical origins of
the clusters.
We consider three qualitatively different scenarios, broadly outlined as:
\begin{enumerate}
  \item The globular cluster system is in place at redshift $z=3$ with orbits
    drawn from a spherical, isotropic distribution function (\acronym{DF}) and
    evolves to present day.
  \item GCs are accreted with a constant accretion rate from redshift $z=3$ to
    present with a spherical, isotropic \acronym{DF}.
  \item GCs are accreted cosmologically from tidally-disrupted satellite
    galaxies.
\end{enumerate}
For scenario 3, we use orbits of subhalos drawn from the Aquarius simulations
(\citealt{aquarius}) and set the number of GCs in each satellite using a $M_{\rm
halo}$-$M_{\rm GC}$ scaling relation (\citealt{Harris:2015}).
With orbital initial conditions for the GCs (from the three scenarios above), a
mass model for the Milky Way, and an initial mass function for the cluster
masses we can evolve a given GC population forward to present day and compare
the population of surviving clusters with that observed around the Milky Way.
% For each case, we add clusters until we find the initial number of GCs that are
% required to produce the ones that survive until present day.

\subsection{Milky Way mass model} \label{sec:massmodel}

For the gravitational potential of the Milky Way, we use a multi-component
mass-model consisting of a spherical Navarro-Frenk-White (NFW) dark matter halo
(\citealt{Navarro:1996}), a Miyamoto-Nagai disk (\citealt{Miyamoto:1975}), and a
sum of spherical Hernquist profiles (\citealt{Hernquist:1990}) to model the mass
distribution in the central Galaxy (bulge + nucleus, though, the detailed shape
of the nuclear potential should not impact the orbits of the halo clusters we
are interested in):
\begin{eqnarray}
  \Phi_{\rm Halo} &=& \frac{G \, M_{\rm H}}{r_{\rm H}}STUFF
  \\
  \Phi_{\rm Disk} &=& -\frac{G \, M_{\rm D}}{STUFF}
  \\
  \Phi_{\rm Bulge} &=& -\frac{G \, M_{\rm B}}{r + r_{\rm B}}
  \\
  \Phi_{\rm Nucleus} &=& -\frac{G \, M_{\rm N}}{r + r_{\rm N}} \quad .
\end{eqnarray}
This is a relatively simple model for the Milky Way potential, however at the
radii that we are most interested (i.e. in the halo) it is a reasonable model
given the vast uncertainties in the shape and profile of the halo.
We fix the Bulge and Disk parameters following \citealt{Bovy:2014}.
We fit for the halo mass, halo scale radius, nucleus mass, and nucleus scale
radius using standard nonlinear least-squares fitting to a compilation of
Milky Way enclosed-mass measurements (see \citealt{Gnedin??}).
The parameter values are given in \tblname~\ref{tbl:potential-params}; in
\figname~\ref{fig:mass-profile}, we plot the mass enclosed as a function of
radii (blue line) and the data used to fit for the free parameters (black
points).

\begin{floattable}
\begin{deluxetable}{r l}
\tabletypesize{\footnotesize}
\caption{Distributions of physical parameters for transit simulations
\label{tbl:potential-params}}

\tablehead{%
    \colhead{name} & \colhead{value}
}
\startdata
$M_{\rm H}$ & $5.73 \times 10^{11}~\msun$ \\
$r_{\rm H}$ & $15.6~\kpc$ \\
\hline
$M_{\rm D}$ & $6.8 \times 10^{10}~\msun$ \\
$r_{\rm D}$ & $3~\kpc$ \\
$z_{\rm D}$ & $280~\pc$ \\
\hline
$M_{\rm B}$ & $5 \times 10^{9}~\msun$ \\
$r_{\rm B}$ & $1~\kpc$ \\
\hline
$M_{\rm N}$ & $1.98 \times 10^{9}~\msun$ \\
$r_{\rm N}$ & $79~\pc$ \\
\enddata

\end{deluxetable}
\end{floattable}

\begin{figure*}[p]
\begin{center}
\includegraphics[width=\textwidth]{figures/mass-profile.pdf}
\end{center}
\caption{%
\todo{caption}
\todo{Add second panel showing circular velocity curve with data from Andreas.}
\todo{What's going on at small radii? Bug in Gala?}
\label{fig:mass-profile}}
\end{figure*}

\subsection{Globular cluster initial mass distribution} \label{sec:gcmassdist}

\subsection{Globular cluster mass evolution} \label{sec:gcmassevolution}

In all cases, we solve for the mass evolution of the clusters using the
prescription used in \citet{Gnedin:2014}.

For cases where a GC is accreted from a satellite galaxy, we first solve for
the mass-loss of the cluster within the satellite system up until it passes the
tidal radius.
At this point, we begin evolving the mass of the cluster in the tidal field of
the Milky Way instead.
When generating mock stellar streams for the GCs, we only consider stars that
are disrupted after the GC is stripped from its parent.

\section{Cosmological subhalo orbits} \label{sec:aqorbits}

\section{Results}

\subsection{}

\section{Conclusions}\label{sec:conclusions}

\acknowledgements
This research made use of
Astropy, a community-developed core Python package for Astronomy
\citep{Astropy-Collaboration:2013}.
% This work used the Extreme Science and Engineering Discovery Environment
% \citep[XSEDE;][]{Towns:2014}, which is supported by National Science Foundation
% grant number ACI-1053575.

\bibliographystyle{aasjournal}
\bibliography{mwgcs}

\end{document}
