%\documentclass[numberedappendix]{emulateapj}
\documentclass[letterpaper,12pt,preprint]{aastex}

% Load common packages

\usepackage{amsmath}
\usepackage{amsfonts}
\usepackage{amssymb}
\usepackage{booktabs}

\usepackage{graphicx}
\usepackage{color}

\usepackage{hyperref}
\definecolor{linkcolor}{rgb}{0,0,0.2}
\hypersetup{colorlinks=true,linkcolor=linkcolor,citecolor=linkcolor,
            filecolor=linkcolor,urlcolor=linkcolor}
\hypersetup{pageanchor=false}

\newcommand{\documentname}{\textsl{Article}}
\newcommand{\sectionname}{Section}
\newcommand{\figname}{Figure}
\newcommand{\eqname}{Equation}
\newcommand{\tblname}{Table}

% Packages / projects / programming
\newcommand{\package}[1]{\textsl{#1}}
\newcommand{\acronym}[1]{{\small{#1}}}
\newcommand{\project}[1]{\package{#1}}
\newcommand{\rewinder}{\project{Rewinder}}
\newcommand{\streakline}{\project{Streakline}}
\newcommand{\superfreq}{\project{SuperFreq}}

\newcommand{\github}{\project{GitHub}}
\newcommand{\python}{\project{Python}}

% For referee
\newcommand{\changes}[1]{{\color{red} #1}}

% Stats / probability
\newcommand{\given}{\,|\,}
\newcommand{\norm}{\mathcal{N}}

% Maths
\newcommand{\dd}{\mathrm{d}}
\newcommand{\transpose}[1]{{#1}^{\mathsf{T}}}
\newcommand{\inverse}[1]{{#1}^{-1}}
\newcommand{\argmin}{\operatornamewithlimits{argmin}}
\newcommand{\mean}[1]{\left< #1 \right>}

% Unit shortcuts
\newcommand{\msun}{\ensuremath{\mathrm{M}_\odot}}
\newcommand{\kms}{\ensuremath{\mathrm{km}~\mathrm{s}^{-1}}}
\newcommand{\pc}{\ensuremath{\mathrm{pc}}}
\newcommand{\kpc}{\ensuremath{\mathrm{kpc}}}
\newcommand{\kmskpc}{\ensuremath{\mathrm{km}~\mathrm{s}^{-1}~\mathrm{kpc}^{-1}}}

% Misc.
\newcommand{\bs}[1]{\boldsymbol{#1}}

% Astronomy
\newcommand{\DM}{{\rm DM}}
\newcommand{\feh}{\ensuremath{{[{\rm Fe}/{\rm H}]}}}
\newcommand{\df}{\acronym{DF}}

% TO DO
\newcommand{\todo}[1]{{\color{red} TODO: #1}}


\begin{document}

\title{The stellar halo of the Milky Way and globular cluster debris I: predictions for kinematic structure}
\author{Adrian~M.~Price-Whelan\altaffilmark{\pu,\adrn},
            Oleg Gnedin\altaffilmark{\umich},
            Kathryn V. Johnston\altaffilmark{\colum},
            David N. Spergel\altaffilmark{\cca,\pu}
%            Hans-Walter~Rix\altaffilmark{\mpia}
}

% Affiliations
\newcommand{\pu}{1}
\newcommand{\adrn}{2}
\newcommand{\umich}{3}
\newcommand{\colum}{4}
\newcommand{\cca}{5}
%\newcommand{\mpia}{6}

\altaffiltext{\pu}{Department of Astrophysical Sciences,
                   Princeton University, Princeton, NJ 08544, USA}
\altaffiltext{\adrn}{To whom correspondence should be addressed:
                     \texttt{adrn@princeton.edu}}

% TODO: michigan
\altaffiltext{\umich}{Department of Astronomy ??,
                      University of Michigan,
                      USA}
\altaffiltext{\colum}{Department of Astronomy,
                      Columbia University,
                      550 W 120th St.,
                      New York, NY 10027, USA}
% TODO: Simons
\altaffiltext{\cca}{Simons Center for Computational Astrophysics,
                      New York, NY XXX, USA}
%\altaffiltext{\mpia}{Max-Planck-Institut f\"ur Astronomie,
%                    K\"onigstuhl 17, D-69117 Heidelberg, Germany}
                     
\begin{abstract}
% Context
Recent work on the chemistry of Milky Way halo stars and the kinematic evolution of globular clusters suggest that (1) a significant fraction of the stellar halo around the Milky Way is composed of disrupted globular clusters, and, consequently, that (2) the present globular cluster population may be a small fraction ($<$5\%) of the initial population.
These suggestions makes strong predictions about the kinematic [...] that depend strongly on the origins of the initial globular cluster system.
% Aims
Here we generate a set of mock stellar halos formed from disrupted globular clusters with an effort to reproduce the present-day kinematics and mass function of the remnant cluster population. 
We then use these halos to predict the expected number of thin stellar streams around the Milky Way and the kinematic clustering of halo stars.
% Methods
We generate the initial radial distribution of the cluster population and solve for the mass-loss histories by [...].
We then generate mock stellar streams with these mass-loss histories with orbital initial conditions either sampled from (1) a Hernquist distribution function with isotropic, radially-anisotropic, or tangentially-anisotropic velocities, or (2) a small number of [...], designed to mimic a scenario in which all of the clusters were accreted early through major mergers.
We ``observe'' the four mock stellar halos with uncertainties expected for the end-of-mission Gaia project.
% Results
We find that within XX kpc XX thin, globular cluster streams by assuming ... BHB/K giants?
We also find that the end-of-Gaia kinematic measurements for XX stars within XX kpc will distinguish the scenarios considered in this work.
Future work will [...] chemical signatures.
\end{abstract}

\keywords{
  Galaxy: halo
  ---
  globular clusters: general
  ---
  stars: kinematics and dynamics
  ---
  Galaxy: structure
  ---
  Galaxy: kinematics and dynamics
}

\section{Introduction}\label{sec:introduction}

\section{Conclusions}\label{sec:conclusions}

\acknowledgements
This research made use of
Astropy, a community-developed core Python package for Astronomy
\citep{Astropy-Collaboration:2013}.
This work used the Extreme Science and Engineering Discovery Environment \citep[XSEDE;][]{Towns:2014}, which is supported by National Science Foundation grant number ACI-1053575.

\bibliographystyle{apj}
\bibliography{refs}

\end{document}
